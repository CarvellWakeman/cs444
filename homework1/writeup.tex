\documentclass[letterpaper,10pt,fleqn]{article}

\usepackage{titling}
\usepackage{url}
\usepackage{enumitem}
\usepackage{geometry}
\geometry{textheight=9.25in, textwidth=6.75in} % 0.75" margins
\usepackage{hyperref}
\usepackage{listings}
\usepackage{cite}

\def\name{Group 26}


\title{Homework 1 Report\\\large CS444 Fall17}
\author{Zach Lerew, Rohan Barve\\\large Group 26}
\date{\today}



\begin{document}

	% Title page
	\begin{titlingpage}
		\maketitle
		\begin{abstract}
			Homework 1 involves sourcing the yocto kernel, building it, and running it with qemu. The second task is to solve a producer/consumer concurrency problem in C.
		\end{abstract}
	\end{titlingpage}


	% Document body
	\section*{Command line qemu parameters}
	\href{https://linux.die.net/man/1/qemu-kvm}{All command descriptions learned from linux man pages}
	\begin{itemize}
		\item-gdb\\
					Wait for gdb connection on device dev. Typical connections will likely be TCP-based

		\item-S\\
					Do not start CPU at startup

		\item-nographic\\
           Normally, if QEMU is compiled with graphical window support, it
           displays output such as guest graphics, guest console, and the QEMU
           monitor in a window. With this option, you can totally disable
           graphical output so that QEMU is a simple command line application.
           The emulated serial port is redirected on the console and muxed with
           the monitor (unless redirected elsewhere explicitly). Therefore, you
           can still use QEMU to debug a Linux kernel with a serial console. Use
           C-a h for help on switching between the console and monitor

	  \item-kernel\\
					The kernel image. Can either be a Linux kernel or in multiboot format.

		\item-drive\\
			     Define a new drive. This includes creating a block driver node (the
			     backend) as well as a guest device, and is mostly a shortcut for
			     defining the corresponding -blockdev and -device options

		\item-enable-kvm\\
					Enable KVM full virtualization support

		\item-net\\
					Indicate that no network devices should be configured. It is used to override the default configuration

		\item-usb\\
           Enable the USB driver (if it is not used by default yet)

		\item-localtime\\
					Sets the time to the current UTC local time

		\item--no-reboot\\
					Exit instead of rebooting

		\item--append\\
					Pass commands to the parameters of the kernel when it is started

	\end{itemize}

	\section*{Kernel build,test,VM Qemu setup}

	\begin{itemize}
		\item mkdir /scratch/fall2017/26.
		\item cd /scratch/fall2017/26.
		\item git clone git://git.yoctoproject.org/linux-yocto-3.19.
		\item cd linux-yocto-3.19.
		\item git checkout \'v3.19.2\', cd ../
		\item cp /scratch/files/environment-setup-i586-poky-linux* ./
		\item cp /scratch/files/bzImage-qemux86.bin ./
		\item cp /scratch/files/core-image-lsb-sdk-qemux86.ext4 ./
		\item source environment-setup-i586-poky-linux
		\item cd linux-yocto-3.19.
		\item make clean.
		\item cp /scratch/files/config-3.19.2-yocto-standard ./.config.
		\item makemenuconfig, adjust general setting Local version to needed group no.
		\item Save the file as .config and exited menuconfig.
		\item Compiled the kernel using: make -j4 bzImage. Used 4 threads.
		\item qemu-system-i386 -gdb tcp::5526 -S -nographic -kernel linux-yocto-3.19/arch/x86/boot/bzImage -drive file=core-image-lsb-sdk-qemux86.ext4,if=virtio -enable-kvm -net none -usb -localtime --no-reboot --append "root=/dev/vda rw console=ttyS0 debug"
		\item Open new terminal session and launch gdb.
		\item Within gdb connect to specified port: (gdb) target remote:5526.
		\item This step should result in the kernel booting and a login screen qemux86 appearing.
		\item The login is root and there is no password so I simply hit Enter.

	\end{itemize}


	\section*{Concurrency assignment}
	\subsection*{Purpose}
	The assignment states that the purpose of concurrency assignments is to hone your skills in thinking in parallel.
	Parallel programming is a large independent topic worthy of study.
	When it comes to operating systems, synchronous work is less common and less useful than parallel async work.
	This assignment helps us practice those skills.
	\subsection*{Approach}
	The problem was fairly easy to solve when you think in terms of objects and analogies.
	\\There are items in a buffer, which must be created by producers, and consumed by consumers.
	\\The buffer can be thought of as a cookie jar with a small opening, only one hand can get in at a time.
	\\Once an item is taken out of the jar, another hand can get into the jar to do its job.
	\\Items can be created and consumed outside of the jar (in parallel), and then fight for a chance to access it when they need to.
	\\Out of this analogy easily comes the pieces needed to solve this problem:
	\begin{itemize}
		\item An item with some data.
		\item A buffer to store those items.
		\item A consumer who locks the mutex, checks the buffer for an item, takes one if it can, and then unlocks the mutex.
		\item A producer who locks the mutex, checks if the buffer is full, adds a new item if it can, and then unlocks the mutex.
		\item And lastly a main function to spawn threads and initialize data.
	\end{itemize}
	\subsection*{Testing}
	With the above approach on paper, a loosely TDD approach was taken.
	\\This problem was small enough however, that a manual test could be used for each step, rather than a traditional failing unit test.
	\\Final results were independently verified by both team members.
	\subsection*{Learned}
	Every time I work with C, I am reminded of its power, as well as its drawbacks.
	This is a fairly simple problem to solve, which makes C an overly complicated tool to solve it with.
	\\In a professional work environment, the tools that can be used are typically limited to the code base owned by the company.
	\\Regardless of the problem and the tools that exist, if the code base is in C\#, the solution should be too (with occasional exceptions).
	\\I understand that the purpose of the class and assignment is low level kernel programming in C (the language of this \'code base\').
	However, why use a 700 bhp v8 in a 4 door sedan that you drive to the grocery store in bumper-to-bumper traffic?

	\section*{Version control log}
		\begin{center}
			\begin{tabular}{ |c|c|c| }
				\hline
				Date & Author & Description \\
				2017-09-30 & Zach Lerew & Initial directory and hw1 start \\
				2017-09-30 & Zach Lerew & Named makefile correctly, updated rules \\
				2017-09-30 & Zach Lerew & Consumer threads created and run correctly \\
				2017-09-30 & Zach Lerew & Consumer threads block until they finish consuming \\
				2017-09-30 & Zach Lerew & Consumer threads wait for buffer items, producer work func made \\
				2017-09-30 & Zach Lerew & Consumers lock and unlock mutex to allow producers a chance to produce \\
				2017-09-30 & Zach Lerew & Implemented random number generation using asm() and Mersenne Twister \\
				2017-10-01 & Zach Lerew & Supporting CS444 syntax preferences \\
				2017-10-01 & Zach Lerew & Added lerewz portion of writeup and latex document base \\
				2017-10-09 & Rohan Barve & Adding modified writeup.tex \\
				2017-10-09 & Rohan Barve & Adding instructions to run kernel for future use \\
				2017-10-09 & Rohan Barve & Added related misc files \\
				2017-10-09 & Zach Lerew & Added git log and abstract to write up \\
				2017-10-09 & Zach Lerew & Removed unnecessary qemu parameters \\
				\hline
			\end{tabular}
		\end{center}

	\section*{Work log}
	This assignment has three distinct sections:
	\begin{itemize}
		\item Kernel building and qemu emulation
		\item Consumer \& Producer concurrency problem
		\item This document
	\end{itemize}
	The work was split between us as such -

	\textbf{Zach}: Concurrency problem, write-up base, makefile, git log

	\textbf{Rohan}: Kernel and VM Qemu setup, configuration and test , concurrency problem review, write-up body

\end{document}
