\documentclass[letterpaper,10pt,fleqn]{article}

\usepackage{titling}
\usepackage{url}
\usepackage{enumitem}
\usepackage{geometry}
\geometry{textheight=9.25in, textwidth=6.75in} % 0.75" margins
\usepackage{hyperref}
\usepackage{listings}
\usepackage{cite}

\def\name{Group 26}


\title{Homework 1 Report\\\large CS444 Fall17}
\author{Zach Lerew, Rohan Barve\\\large Group 26}
\date{\today}



\begin{document}

	% Title page
	\begin{titlingpage}
		\maketitle
		\begin{abstract}
			This is an abstract
		\end{abstract}
	\end{titlingpage}


	% Document body
	\section*{Command line qemu parameters}
	\begin{itemize}
		\item-h Display help and exit
		\item-version Display version information and exit
		\item-machine [type=]name[,prop=value[,...]]
					 Select the emulated machine by name. Use "-machine help" to list
           available machines.
		\item-cpu model
           Select CPU model ("-cpu help" for list and additional feature
           selection)

		\item-accel name[,prop=value[,...]]
           This is used to enable an accelerator. Depending on the target
           architecture, kvm, xen, hax or tcg can be available. By default, tcg
           is used. If there is more than one accelerator specified, the next
           one is used if the previous one fails to initialize.
		\item--smp
       [cpus=]n[,cores=cores][,threads=threads][,sockets=sockets][,maxcpus=maxcpus]
           Simulate an SMP system with n CPUs. On the PC target, up to 255 CPUs
           are supported. On Sparc32 target, Linux limits the number of usable
           CPUs to 4.  For the PC target, the number of cores per socket, the
           number of threads per cores and the total number of sockets can be
           specified. Missing values will be computed. If any on the three
           values is given, the total number of CPUs n can be omitted. maxcpus
           specifies the maximum number of hotpluggable CPUs.
		\item-numa node[,mem=size][,cpus=firstcpu[-lastcpu]][,nodeid=node]
		\item-numa node[,memdev=id][,cpus=firstcpu[-lastcpu]][,nodeid=node]
		\item-numa dist,src=source,dst=destination,val=distance
		\item-numa cpu,node-id=node[,socket-id=x][,core-id=y][,thread-id=z]
           Define a NUMA node and assign RAM and VCPUs to it.  Set the NUMA
           distance from a source node to a destination node
		\item-add-fd fd=fd,set=set[,opaque=opaque]
           Add a file descriptor to an fd set.
		\item-set group.id.arg=value
           Set parameter arg for item id of type group
		\item-global driver.prop=value
		\item-mem-path path
           Allocate guest RAM from a temporarily created file in path
		\item -mem-prealloc
           Preallocate memory when using -mem-path
		\item-k language
           Use keyboard layout language (for example "fr" for French). This
           option is only needed where it is not easy to get raw PC keycodes
           (e.g. on Macs, with some X11 servers or with a VNC or curses
           display). You don't normally need to use it on PC/Linux or PC/Windows
           hosts
		\item-audio-help
           Will show the audio subsystem help: list of drivers, tunable
           parameters
		\item-soundhw card1[,card2,...] or -soundhw all
           Enable audio and selected sound hardware. Use 'help' to print all
           available sound hardware
		\item-balloon none
           Disable balloon device
		\item-balloon virtio[,addr=addr]
           Enable virtio balloon device (default), optionally with PCI address
           addr
		\item-name name
           Sets the name of the guest.  This name will be displayed in the SDL
           window caption.  The name will also be used for the VNC server.  Also
           optionally set the top visible process name in Linux.  Naming of
           individual threads can also be enabled on Linux to aid debugging

		\item-uuid uuid
           Set system UUID

    \item-fda file
		\item-hdd file
           Use file as hard disk 0, 1, 2 or 3 image
		\item-cdrom file
           Use file as CD-ROM image (you cannot use -hdc and -cdrom at the same
           time). You can use the host CD-ROM by using /dev/cdrom as filename

		\item-blockdev option[,option[,option[,...]]]
           Define a new block driver node. Some of the options apply to all
           block drivers, other options are only accepted for a specific block
           driver. See below for a list of generic options and options for the
           most common block drivers

		\item-drive option[,option[,option[,...]]]
           Define a new drive. This includes creating a block driver node (the
           backend) as well as a guest device, and is mostly a shortcut for
           defining the corresponding -blockdev and -device options

		\item-full-screen
           Start in full screen

		\item-nographic
           Normally, if QEMU is compiled with graphical window support, it
           displays output such as guest graphics, guest console, and the QEMU
           monitor in a window. With this option, you can totally disable
           graphical output so that QEMU is a simple command line application.
           The emulated serial port is redirected on the console and muxed with
           the monitor (unless redirected elsewhere explicitly). Therefore, you
           can still use QEMU to debug a Linux kernel with a serial console. Use
           C-a h for help on switching between the console and monitor

		\item-display type
           Select type of display to use. This option is a replacement for the
           old style -sdl/-curses/... options
		\item-usb
           Enable the USB driver (if it is not used by default yet)

		\item-usbdevice devname
           Add the USB device devname. Note that this option is deprecated,
           please use "-device usb-..." instead
		\item -full-screen
           Start in full screen
		\item-vnc display[,option[,option[,...]]]
           Normally, if QEMU is compiled with graphical window support, it
           displays output such as guest graphics, guest console, and the QEMU
           monitor in a window. With this option, you can have QEMU listen on
           VNC display display and redirect the VGA display over the VNC
           session. It is very useful to enable the usb tablet device when using
           this option (option -device usb-tablet). When using the VNC display,
           you must use the -k parameter to set the keyboard layout if you are
           not using en-us

	\end{itemize}

	\section*{Kernel build,test,VM Qemu setup}

	\begin{itemize}
		\item mkdir /scratch/fall2017/26.
		\item cd /scratch/fall2017/26.
		\item git clone git://git.yoctoproject.org/linux-yocto-3.19.
		\item cd linux-yocto-3.19.
		\item git checkout 'v3.19.2', cd ../
		\item cp /scratch/files/environment-setup-i586-poky-linux* ./
		\item cp /scratch/files/bzImage-qemux86.bin ./
		\item cp /scratch/files/core-image-lsb-sdk-qemux86.ext4 ./
		\item source environment-setup-i586-poky-linux
		\item cd linux-yocto-3.19.
		\item make clean.
		\item cp /scratch/files/config-3.19.2-yocto-standard ./.config.
		\item makemenuconfig, adjust general setting Local version to needed group no.
		\item Save the file as .config and exited menuconfig.
		\item Compiled the kernel using: make -j4 bzImage. Used 4 threads.
		\item qemu-system-i386 -gdb tcp::5526 -S -nographic \
    -kernel linux-yocto-3.19/arch/x86/boot/bzImage \
    -drive file=core-image-lsb-sdk-qemux86.ext4,if=virtio \
    -enable-kvm -net none -usb -localtime --no-reboot \
    --append "root=/dev/vda rw console=ttyS0 debug"
		\item Open new terminal session and launch gdb.
		\item Within gdb connect to specified port: (gdb) target remote:5526.
		\item This step should result in the kernel booting and a login screen qemux86 appearing.
		\item The login is root and there is no password so I simply hit Enter.

	\end{itemize}


	\section*{Concurrency assignment}
	\subsection*{Purpose}
	The assignment states that the purpose of concurrency assignments is to hone your skills in thinking in parallel.
	Parallel programming is a large independent topic worthy of study.
	When it comes to operating systems, synchronous work is less common and less useful than parallel async work.
	This assignment helps us practice those skills.
	\subsection*{Approach}
	The problem was fairly easy to solve when you think in terms of objects and analogies.
	\\There are items in a buffer, which must be created by producers, and consumed by consumers.
	\\The buffer can be thought of as a cookie jar with a small opening, only one hand can get in at a time.
	\\Once an item is taken out of the jar, another hand can get into the jar to do its job.
	\\Items can be created and consumed outside of the jar (in parallel), and then fight for a chance to access it when they need to.
	\\Out of this analogy easily comes the pieces needed to solve this problem:
	\begin{itemize}
		\item An item with some data.
		\item A buffer to store those items.
		\item A consumer who locks the mutex, checks the buffer for an item, takes one if it can, and then unlocks the mutex.
		\item A producer who locks the mutex, checks if the buffer is full, adds a new item if it can, and then unlocks the mutex.
		\item And lastly a main function to spawn threads and initialize data.
	\end{itemize}
	\subsection*{Testing}
	With the above approach on paper, a loosely TDD approach was taken.
	\\This problem was small enough however, that a manual test could be used for each step, rather than a traditonal failing unit test.
	\\Final results were independently verified by both team members.
	\subsection*{Learned}
	Every time I work with C, I am reminded of its power, as well as its drawbacks.
	This is a fairly simple problem to solve, which makes C an overly complicated tool to solve it with.
	\\In a professional work environment, the tools that can be used are typically limited to the code base owned by the company.
	\\Regardless of the problem and the tools that exist, if the code base is in C\#, the solution should be too (with occasional exceptions).
	\\I understand that the purpose of the class and assignment is low level kernel programming in C (the language of this 'code base').
	However, why use a 700 bhp v8 in a 4 door sedan that you drive to the grocery store in bumper-to-bumper traffic?

	\section*{Version control log}
	Git repo log

	\section*{Work log}
	This assignment has three distinct sections:
	\begin{itemize}
		\item Kernel building and qemu emulation
		\item Consumer \& Producer concurrency problem
		\item This document
	\end{itemize}
	The work was split between us as such -

	\textbf{Zach}: Concurrency problem, document base / makefile

	\textbf{Rohan}: Kernel and VM Qemu setup, configuration and test , concurrency problem review, document body

\end{document}
