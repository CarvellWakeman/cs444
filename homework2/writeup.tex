\documentclass[letterpaper,10pt,fleqn]{article}

\usepackage{titling}
\usepackage{url}
\usepackage{enumitem}
\usepackage{geometry}
\geometry{textheight=9.25in, textwidth=6.75in} % 0.75" margins
\usepackage{hyperref}
\usepackage{listings}
\usepackage{cite}

\def\name{Group 26}


\title{Homework 2 Report\\\large CS444 Fall17}
\author{Zach Lerew, Rohan Barve\\\large Group 26}
\date{\today}



\begin{document}

	% Title page
	\begin{titlingpage}
		\maketitle
		\begin{abstract}
			\noindent Homework 2 involves modifying the linux-yocto kernel to use a different I/O scheduler algorithm.
			The algorithm this team chose to use was the C-LOOK algorithm due to its simplicity.
			This assignment involved modifying the kernel at a very low level, and learning how to rebuild and run it, and test that the changes worked.
		\end{abstract}
	\end{titlingpage}


	% Document body
	\section*{CLOOK algorithm design}
	The C-LOOK I/O scheduling algorithm is rather simple, but solves an issue presented by algorithms such as FIFO. When reading or writing with a hard disk drive, there is a physical head that must move to the correct sector on the spinning platter.
	This movement takes time, so it should be done as little as possible. The C-LOOK algorithm takes incoming requests and sorts them into a queue in an increasing order.
	As the system operates, the RW head starts at the lowest (most inner) sector of the queue, and moves in an increasing order towards the outer edge of the platter.
	When it reaches the last (largest) element in the queue, it moves directly back to the inner edge of the disk to begin the process again.
	\\After researching this issue, we came to a realization we believe to be true. There are two primary ways to accomplish the task of dispatching requests in a least to greatest order in the C-LOOK algorithm:
	\subsection{Sort twice in dispatch}
	Add requests to the end of the queue regardless of the location of the request, or the position of the head. When a request is dispatched, the queue is sorted so that the nearest request is serviced first. Then once a request has been removed from the queue, sort again.
	\subsection{Sort in dispatch and in add}
	Use an insertion sort to add requests to the queue, and sort the queue once a sector has been dispatched.
	This is the option our team took, primarily because the runtime of insertion sort is known and controllable compared to calling a kernel sort function with an unknown implementation.


	\section*{Questions}
	\subsection*{1.What do you think the main point of this assignment is?}
	Understand how I/O schedulers work in a kernel environment.
	Replace the NO-OP scheduler in the yocto-linux block directory with an implementation of the C-LOOK scheduler.
	Test and verify our implementation of the C-LOOK I/O scheduler, and write clear instructions for the TA grading the assignment.

	\subsection*{2.How did you personally approach the problem? Design decisions, algorithm, etc.}
	The first step in this process was to understand how the C-LOOK algorithm works from a conceptual point of view.
	After that, the goal was to copy the NO-OP scheduler source code into a new file called \textit{sstf-iosched.c}.
	Next, the Kconfig.iosched file and Makefile in the block directory had to be modified to use the NO-OP scheduler copy.
	After the kernel was successfully using a new I/O scheduler, the algorithm could be implemented in code.
	The last step was to test that the algorithm was working as expected.

	\subsection*{3.How did you ensure your solution was correct? Testing details, for instance.}
	To verify that the C-LOOK algorithm is working as expected, \textit{printk} statements were placed in every function in the scheduler.
	As files are written or read from the disk, the CLOOK scheduler will print to \textbf{stdout}.
	These statements can be seen as the kernel is running, or when the \textit {dmseg} command is run.
	We were able to verify that the algorithm was working as expected by looking at the calls to the dispatch function, and observing that the sector number was always increasing.
	Additionally, the NO-OP scheduler uses a FIFO queue, meaning that sector requests are added and dispatched immediately.
	In the C-LOOK scheduler, several sector requests are made in a row, and dispatch is called opportunistically in order by sector.

	\subsection*{4.What did you learn?}
	I/O scheduling is an important part of a Unix kernel, but most of the work is abstracted into files like \textit{linux\\elevator.h}.
	The \textit{Kconfig} file specifies which set of functions should be used for the actions like merge, add, and dispatch.
	This allows different algorithms to be used for I/O scheduling without requiring a massive amount of effort.
	The C-LOOK algorithm slightly improves efficiency over algorithms like NO-OP by allowing the drive head to move the least amount possible.

	\subsection*{5.How should the TA evaluate your work? Provide detailed steps to prove correctness.}
	\begin{itemize}
		\item Apply the kernel patch file and build the kernel.
		\item The build process should ask you to specify which IOscheduler to use, select CLOOK and continue building.
		\item Source the correct environment file for your shell. This assignment was made using bash.
		\item Launch qemu using the following command:
		qemu-system-i386 -gdb tcp::5526 -S -nographic -kernel ./linux-yocto-3.19/arch/x86/boot/bzImage -drive file=core-image-lsb-sdk-qemux86.ext4,if=ide -enable-kvm -usb -localtime --no-reboot --append "root=/dev/hda rw console=ttyS0 debug"

		\item After the kernel boots (use gdb to continue execution), you should see several CLOOK print statements as confirmation that the new scheduler is being used.
		\item Navigate to the root directory with \textit{cd /}
		\item Run the following command while observing the output statements from CLOOK: \textit{mkdir ./cpy \&\& cp -r ./* ./cpy}
		\item Notice the output print statements - there should be consecutive groups of add requests, interspersed with dispatch statements.
		The dispatch statements should always be in increasing order, but are not required to come directly after the add requests.
	\end{itemize}

	\section*{Version control log}
	\begin{center}
				\begin{tabular}{ |c|c|c| }
					\hline
					Date & Author & Description \\
					\hline
					2017-10-12 & Zach Lerew & Initial directory and hw2 files copied from hw1 \\
					2017-10-12 & Zach Lerew & Added gitignore, writeup sections, removed hw1 code \\
					2017-10-19 & Zach Lerew & Added noop base code from block directory \\
					2017-10-20 & Zach Lerew & Copied noop base into sstf-iosched.c \\
					2017-10-20 & Zach Lerew & Replaced noop references with clook \\
					2017-10-20 & Rohan Barve & Modified Kconfig to use a custom scheduler, reorganized directory \\
					2017-10-21 & Rohan Barve & Added print statements for testing and debugging \\
					2017-10-22 & Zach Lerew & Modified clook\_add function to insert sort requests to the correct place in the queue \\
					2017-10-22 & Zach Lerew & Added comments to source code and cleaned up\\
					2017-10-22 & Rohan Barve & Moved design descriptions from team meeting scratch notes to writeup \\
					2017-10-23 & Rohan Barve & Answered questions in writeup \\
					2017-10-23 & Zach Lerew & Added version control log to writeup \\
					2017-10-24 & Rohan Barve & Added kernel patch file \\

					\hline
				\end{tabular}
			\end{center}

\end{document}
