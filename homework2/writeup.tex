\documentclass[letterpaper,10pt,fleqn]{article}

\usepackage{titling}
\usepackage{url}
\usepackage{enumitem}
\usepackage{geometry}
\geometry{textheight=9.25in, textwidth=6.75in} % 0.75" margins
\usepackage{hyperref}
\usepackage{listings}
\usepackage{cite}

\def\name{Group 26}


\title{Homework 2 Report\\\large CS444 Fall17}
\author{Zach Lerew, Rohan Barve\\\large Group 26}
\date{\today}



\begin{document}

	% Title page
	\begin{titlingpage}
		\maketitle
		\begin{abstract}
			\noindent Homework 2
		\end{abstract}
	\end{titlingpage}


	% Document body
	\section*{CLOOK algorithm design}
	The C-LOOK I/O scheduling algorithm is rather simple, but solves an issue presented by algorithms such as FIFO. When reading or writing with a hard disk drive, there is a physical head that must move to the correct sector on the spinning platter.
	This movement takes time, so it should be done as little as possible. The C-LOOK algorithm takes incoming requests and sorts them into a queue in an increasing order.
	As the system operates, the RW head starts at the lowest (most inner) sector of the queue, and moves in an increasing order towards the outer edge of the platter.
	When it reaches the last (largest) element in the queue, it moves directly back to the inner edge of the disk to begin the process again.
	\\After researching this issue, we came to a realization we believe to be true. There are two primary ways to accomplish the task of dispatching requests in a least to greatest order in the C-LOOK algorithm:
	\subsection{Sort twice in dispatch}
	Add requests to the end of the queue regardless of the location of the request, or the position of the head. When a request is dispatched, the queue is sorted so that the nearest request is serviced first. Then once a request has been removed from the queue, sort again.
	\subsection{Sort in dispatch and in add}
	Use an insertion sort to add requests to the queue, and sort the queue once a sector has been dispatched.
	This is the option our team took, primarily because the runtime of insertion sort is known and controllable compared to calling a kernel sort function with an unknown implementation.

	
	\section*{Questions}
	\subsection*{1.What do you think the main point of this assignment is?}
	\subsection*{2.How did you personally approach the problem? Design decisions, algorithm, etc.}
	\subsection*{3.How did you ensure your solution was correct? Testing details, for instance.}
	\subsection*{4.What did you learn?}
	\subsection*{5.How should the TA evaluate your work? Provide detailed steps to prove correctness.}

	\section*{Version control log}


	\section*{Work log}

	\textbf{Zach}: Work details
	\\\textbf{Rohan}: Work details

\end{document}
